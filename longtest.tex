\documentclass[12pt]{article}
\usepackage{fullpage}
\usepackage{setspace}

\usepackage{oobib}

%\clubpenalty=300
%\widowpenalty=300

\NEW{Chapter}{\cHopGlob}{
   \s\MYauthors  {\name{Ian Condry}}
   \s\MYtitle    {Japanese Hip-Hop and the Globalization of Popular Culture}
   \s\MYbook     {Urban Life, 4th Edition}
   \s\MYcity     {Prospect Heights, IL}
   \s\MYpublisher{Waveland Press}
   \s\MYyear     {2001}
   \s\MYpages    {357--387}
 }
 
 \NEW{Article}{\cPiracy}{
   \s\MYauthors  {\name{Ian Condry}}
   \s\MYtitle    {Cultures of Music Piracy: An Ethnographic Comparison of the US and Japan}
   \s\MYjournal  {International Journal of Cultural Studies}
   \s\MYvolume   {7}
   \s\MYnumber   {3}
   \s\MYyear     {2004}
   \s\MYpages    {343--363}
%   \s\MYpublisher{Sage Publications}
   }
 
 \NEW{Chapter}{\cChing}{
   \s\MYauthors  {\name{Leo Ching}}
   \s\MYtitle    {Imaginings in the Empires of the Sun: Japanese Mass Culture in Asia}
   \s\MYbook     {Contemporary Japan and Popular Culture}
   \s\MYeditors  {\name{John Whittiertreat}}
   \s\MYcity     {Honolulu}
   \s\MYpublisher{University of Hawai'i Press}
   \s\MYyear     {1996}
   \s\MYpages    {169--194}
   }
   
 \NEW{Chapter}{\cPark}{
   \s\MYauthors  {\name{Jung-Sun Park}}
   \s\MYtitle    {Korean American Youths' Consuption of Korean and Japanese TV Dramas and Its Implications}
   \s\MYbook     {Feeling Asian Modernities: Transnational Consumption of Japanese TV Dramas}
   \s\MYeditors  {\name{Koichi Iwabuchi}}
   \s\MYcity     {Hong Kong}
   \s\MYpublisher{Honk Kong University Press}
   \s\MYyear     {2004}
   \s\MYpages    {275--300}
   }
 
 \NEW{Chapter}{\cBlues}{
   \s\MYauthors  {\name{E. Taylor Atkins}}
   \s\MYtitle    {Can Japanese Sing the Blues?: `Japanese Jazz' and the Problem of Authenticity}
   \s\MYbook     {Japan Pop!: Inside the World of Japanese Popular Culture}
   \s\MYeditors  {\name{Timothy J. Craig}}
   \s\MYcity     {Armonk, New York}
   \s\MYpublisher{M.\ E.\ Sharpe}
   \s\MYyear     {2000}
   \s\MYpages    {27--59}
   }

\NEW{Article}{\cCool}{
  \s\MYauthors  {\name{Douglas McGray}}
  \s\MYtitle    {Japan's Gross National Cool}
  \s\MYjournal  {Foreign Policy}
  \s\MYissue    {130}
  \s\MYmonth    {May/June}
  \s\MYyear     {2002}
  \s\MYpages    {45--54}
  }

\NEW{Chapter}{\cSurface}{
  \s\MYauthors  {\name{Botond Bognar}}
  \s\MYtitle    {Surface above All?: American Influence on Japanese Urban Space}
  \s\MYbook     {Transactions, Transgressions, Transformations: American Culture in Western Europe and Japan}
  \s\MYeditors  {\name{Heide Fehrenbach}%
                 \name{Uta G.\ Poiger}}
  \s\MYcity     {New York}
  \s\MYpublisher{Berghahn Books}
  \s\MYyear     {2000}
  \s\MYpages    {45--78}
  }

\NEW{Chapter}{\cWaiting}{
  \s\MYauthors  {\name{Takayuki Tatsumi}}
  \s\MYtitle    {Waiting for Godzilla: Chaotic Negotiations between Post-Orientalism and Hyper-Occidentalism}
  \s\MYbook     {Transactions, Transgressions, Transformations: American Culture in Western Europe and Japan}
  \s\MYeditors  {\name{Heide Fehrenbach}%
  	         \name{Uta G.\ Poiger}}
  \s\MYcity     {New York}
  \s\MYpublisher{Berghahn Books}
  \s\MYyear     {2000}
  \s\MYpages    {224--236}
  }

\NEW{Book}{\cRecentering}{
  \s\MYauthors  {\name{Koichi Iwabuchi}}
  \s\MYtitle    {Recentering globalization: Popular culture and Japanese transnationalism}
  \s\MYcity     {Durham, North Carolina}
  \s\MYpublisher{Duke University Press}
  \s\MYyear     {2002}
  }

\NEW{Book}{\cEncyl}{
  \s\MYauthors  {\name{Mark Schilling}}
  \s\MYtitle    {The Encyclopedia of Japanese Pop Culture}
  \s\MYcity     {New York}
  \s\MYpublisher{Weatherhill}
  \s\MYyear     {1997}
  }


\title{Cultural Exchange\\{\Large Reinterpretation of Foreign Ideas in Japan and the United States}}
\author{Xavid Pretzer}
\date{\today}

\begin{document}

\maketitle

\doublespacing

\section{Introduction}

%In a world with instantaneous communication and the rapid spread of ideas, culture has begun to evolve in a way that defies traditional national classifications.  While various countries have influenced each other's cultures strongly in past eras, it was always in a slower and more deliberate manner, not the hectic uncontrolled rush it may seem like today.

Much has been made of the cultural risks of globalization, the idea that ``cultural imperialism'' is destroying national uniqueness in a world where political boundaries restrict the flow of ideas much less. 
The spread of McDonald's and Hollywood worldwide to the detriment of local cuisine and traditional entertainment seems to some to be replacing the distinct traditional cultures of the world with a single, monotonic, Americanized culture.  However, in looking at the cultural effects of globalization, it is important to examine the details and actual perceptions of cultural elements and not judge cultural change purely by surface appearances.  Excellent examples of the subtleties of cultural influence are the cases of Japan and the United States.  In the past 150 years, Japan has faced huge internal and external pressures to accept and assimilate Western ideas and attitudes.  In addition, its economic strength has given the Japanese people much greater opportunities to emulate the American lifestyle than most other non-Western nations.  However, despite these forces, modern Japanese culture has not become a shallow American imitation, and in fact shows strong signs of keeping both traditional characteristics and unique interpretations of foreign-derived ideals.  Conversely, if globalization were creating a single Americanized global culture, certainly one would expect the culture of the United States to fully embrace this paradigm.  However, American culture itself shows signs of increased global influence, both in mainstream culture and in subgroups such as the Japan-focused \emph{otaku}.  While these influences may be less obvious than the spread of Starbucks abroad, they are none-the-less real and paint a more optimistic picture of the potentials for world culture under globalization.
Indeed, by looking at the effects of American culture in Japan and Japanese influences in the United States, one can see a plausible scenario for long-term cultural behavior in modernized, open societies.  These two mirrored examples show both how long-term exposure to foreign cultures can influence without obliterating local distinctiveness and how globalization, while blurring the lines between national cultures in some ways, also helps to create new subcultures based on other boundaries. 

\section{A History of Absorption}

Japan has a unique history of absorbing aspects of foreign cultures without losing its own traditions.  Before the modern era, Japan went through several cycles where it would rapidly import ideas such as Buddhism and writing from China and Korea and them isolate itself from the world and create uniquely Japanese interpretations of these imported ideas.~\cSurface{\citep[pg.\ 48]}
%Stuff about Japan historically importing and Japanizing culture~\citep[pg.\ 48]{surface}
In fact, it can be argued that almost all of traditional Japanese culture today shows strong oversea influences, reminding us that even in past eras, the idea of separate, pure national cultures was hardly accurate.  When Japan first encountered Europeans in the 16th century, it initially treated them in much the same way, fascinated by these strange-looking people and the wondrous tools and ideas they brought.  However, the ruling Tokugawa Shogunate began to fear the influence of the Europeans and the spread of Christianity, and by 1641 had, with the exception of a small section of Nagasaki, isolated Japan from the outside world once again, preserving Tokugawa power and Japanese traditions but keeping Japan ignorant of the great changes occurring throughout the world.

When Commodore Perry forced Japan to open itself to American trade in 1854, Japan had been in one of these periods of isolation, and thus was mostly unaware of world developments for the past 200 years.  While there was initially some resistance to Western influence, after the Meiji Restoration in 1869, Japan began to embrace modernization and Western culture the way it had pursued Chinese ideas in the past.
Unlike almost all of the non-Western world, Japan was never colonized by a foreign power, nor was it ever conquered from the time the ancestors of today's Japanese people arrived on the island until World War II.  While it's opening to trade was forced, it's process of Westernization was much more self-motivated, and thus less-oppressive, than that of most non-Western countries.  Japan adopted Western practices not because it was forced to but because it saw the power the United States and European nations wielded in the world, and it wanted to assert its place among them.  When the Meiji regime came into power, with it came great impetus to modernize the government and its institutions.  The Emperor moved to the city of Edo, now rechristened T\=oky\=o, and foreign experts were brought in to modernize government institutions, military practices, and the city itself.

\section{Japanese Interpretations}

Today, it is impossible to miss the influence of American institutions on the Japanese landscape.  With more American food chains and Hollywood movies than any country outside North America, Japan is ``still plugged into every trend of Western, especially American, pop culture''.~\cChing{\citep[pg.\ 171]}
% ``Despite its waning economic presence, American mass culture is still the dominant and powerful cultural industry in the world today. \ldots{} Japanese mass culture itself is still plugged into every trend of Western, especially American, pop culture.  Today Japan has more American food franchises than any country outside of North America, and Japan also serves as the largest market for Hollywood films outside the United States.''~\citep[pg.\ 171]{ching}
While Japanese traditions and institutions still persist in diverse areas, American fashion, celebrities, and cartoon characters enjoy huge popularity among the youth, and images of foreigners are common in advertising for even Japanese products.  With the prevalence of nonsensical English in shop names and song lyrics, it is easy to dismiss this as a blind obsession characterized by abandonment of Japanese tradition in favor of shallow imitations.  However, beneath the surface there is a lot more Japan and a lot less blind copying than it would seem.

% Japanese want Americanization
One key difference between the so-called ``cultural imperialism'' facing the world today and the political and economic imperialisims of previous eras is that the spread of American culture is not only voluntary, it is often in great demand by those who receive it, and this is certainly true in Japan.  As \cBlues{\citet[pg.\ 28]} writes, 
``While there has always been resistance to political and cultural domination by the United States \ldots{} the American presence in Japanese cultural life usually has been accepted, even welcomed, by consumers.  In the democracy of the marketplace, America is a clear winner.''  While some members of the older generations oppose the spread of American influences, the youth, especially young women, embrace Western cultures wholeheartedly, taking extra English classes and hoping to study or work overseas.  While the worry that cultural distinctiveness will be lost only to be missed when it is too late to resurrect is legitimate, the voluntary nature of this cultural spread makes it harder to directly oppose.

% Is Americanization done?
An important issue when looking at Americanization is the rate and extent at which it is occurring.  Many analysts critical of Americanization seem to assume that if measures aren't taken to prevent it, the spread of American culture will continually accelerate until it has entirely replaced local distinctiveness.  However, the example of Japan, a country with great pressure for Americanization and few obstacles, casts some doubt on this idea.  For example, \cHopGlob{\citet[pg.\ 385]} notes in his study of Japanese hip-hop that, while after the war Western music sales greatly exceeded sales of native music, today they make up only one-fourth of the Japanese music market.
%``In the immediate postwar period in Japan, Western music initially dominated sales.  But sales of Japanese music steadily grew and in 1967 outpaced Western sales.  Today, three-fourths of Japan's music market is Japanese music to one-fourth Western.''~\citep[pg.\ 385]{condry:hopglob}
Part of this shift is of course due to Japan's postwar industrialization, which gave it better ability to produce local music; nevertheless, the fact that consumers given the choice prefer Japanese songs shows that American influenced has not homogenized the music scene. Still, a cultural purist would argue that the three-fourths that are Japanese are so dominated by songs in foreign styles that they might as well be American imports.  However, just because a song is in an American style doesn't necessarily mean it is not based on Japanese cultural values.

% Does surface => values?
A question that must be addressed to understand Americanization is to what extent the popularity of American products actually represents a change in cultural values and not just a more varied marketplace.  While critics of globalization are prone to view any spread of American pop culture as destructive, Ian Condry believes that
%``Many people view globalization, and particularly the spread of American pop culture, as a \ldots{} kind of invasion, but 
``the idea that watching a Disney movie automatically instills certan values must be examined and not simply assumed.''~\cHopGlob{\citep[pg.\ 382]}
Indeed, if the adoption of American images is shallow copying driven by the allure of the exotic, as some claim, it seems reasonable that it would have little effect on the deeper bases of Japanese culture.  In fact, just as many ethnic restaurants in the United States modify their cuisine to suit local tastes, American imports from fast food chains to clothing styles are adjusted to fit Japanese tastes and preferences, not simply overriding them.  Condry later states,
``Just as it would seem strange to Americans if someone claimed Pok\'emon is making U.S.\ kids `more Japanese,' it is dangerous to assume that mass culture goods by themselves threaten to overwhelm other cultures.''~\cHopGlob{\citep[pg.\ 386]}  So, while American institutions in Japan do have a significant cultural influence, it is easy to overestimate the fallout based on initial appearances.

% Interpretations are unique
Thus, it can be seen that while American influence is certainly omnipresent in modern Japan, beneath the surface Japan retains its own uniqueness.  As in the past, but without the need for isolation, Japan takes in foreign ideas and then interprets them to make them its own.  In considering Japan's cities, with their Western-style office buildings and frequent convenience stores, one researcher writes,
``While in many areas some, and in some areas most, of the constituent elements are of American (or other foreign) origin, the overall spatial matrix or organizing system, as well as the semantic field, are unlike the American or any other Western model.''~\cHopGlob{\citep[pg.\ 73]}
This shows how the Japanese people, while enamored with American elements, do not blindly copy them, but adapt them to their own cultural framework.  Japanese convenience stores sell ties and shirts for businessmen who missed the last train home and allow you to pay for online purchases; in a cash-dominated society, some Japanese online stores don't take credit cards.  Many of Tokyo's buildings, from department stores to Tokyo Tower, the highest man-made structure in Japan, have small Shint\=o shrines on the roof.  Even in the wild atmosphere of Shibuya's clubs, friends still give each other traditional formal benedictions on New Year's Day.~\cHopGlob{\citep[pg.\ 380]}  In modern art as well, while Japanese artists initially predominantly imitated Western styles and works, ``their art of `mimicry' has domesticated and even outgrown the Wester other.  They paved the way \ldots{} for the highly chaotic and splendidly creative negotiations between Western and non-Western cultures that have recently occured.''~\cWaiting{\citep[pg.\ 233]}
%``While non-Western artists started their careers by imitating Western works of art, their art of `mimicry' has domesticated and even outgrown the Wester other.  They paved the way not only for the late capitalist synchronicity between different cultures, but also for the highly chaotic and splendidly creative negotiations between Western and non-Western cultures that have recently occured.''~\citep[pg.\ 233]{waiting}
It is clear that, once you get beneath the surface, the initial impressions of shallow copying are replaced with unique and distinctive Japanese interpretations of foreign-derived elements.

% Take the case of Japanese hip-hop
One example of a Japanese subculture that would appear to simply be an imitation of an American subculture would be Japan's hip-hop movement.  The clothing, rapping, and graffiti look very familiar to American visitors and seem entirely at odds with the values of Japanese culture.  Stores and clubs that target the hip-hop crowd even hire black foreigners to advertise and create an impression of `authenticity' that seems ridiculous to outside observers.  However, in Ian Condry's studies of the Japanese hip-hop movement, he presents a much deeper picture, showing that by looking at the actual source locations for the Japanese movement, it can be seen that this is not simple idolization or copying but actually a process by which a foreign institution is transformed to reflect local conditions and characteristics.~\cHopGlob{\citep[pg.\ 381]}  He found that Japanese hip-hop groups get their start not by directly imitating American hip-hop artists, but through freestyling at a club in front of other hip-hop artists.%~\citep[pg.\ 379]{condry:hopglob}  
This means that rather than constantly imitating the American style, Japanese hip-hop artists recognize that they have something legitimate and worth-while themselves, and work to develop and improve that within the group rather than taking from outside.  Established hip-hop groups with similar styles will band together into `families' that ensure unique and different sounds within the Japanese hip-hop scene and also allow different artists to support each other creatively.~\cHopGlob{\citep[pg.\ 379]}  The lyrics, too, often focus on Japanese themes from traditional stories to video games.  
One should not overemphasize the level of conformity of hip-hop with mainstream Japanese principles; in Japan's group-oriented context, ``the hip-hop idea that one should be speaking for oneself is, in some limited sense, revolutionary.''~\cHopGlob{\citep[pg.\ 384]}
%``Moreover, the dominant ideology that harmony of the group should come before individual expression \ldots{} makes for a social context in which the hip-hop idea that one should be speaking for oneself is, in some limited sense, revolutionary.''~\citep[pg.\ 384]{condry:hopglob}
Nevertheless, examining the source of Japanese hip-hop shows a distinctive and entirely Japanese interpretation of what could otherwise be an entirely American genre.

%``Yet if we are to understand the shape of cultural forms in a world that is increasingly connected by global media and commodity flows, we must situate Japanese rappers in the context of contemporary Japan.''~\citep[pg.\ 374]{condry:hopglob}

% Subcultures => conclusion
The example of Japanese hip-hop shows that, even in a world where American culture and dominant companies seem to be gaining more and more cultural influence, ``there is an equally profound if less visible process by which niche scenes are becoming deeper and more widely connected than before, and in the process, new forms of heterogeneity are born.''~\cHopGlob{\citep[pg.\ 385]}
%``On one hand, the recording industry is reaching larger and larger markets, both within Japan and around the world, as mega-hits continue to set sales records.  On the other hand, there is an equally profound if less visible process by which niche scenes are becoming deeper and more widely connected than before, and in the process, new forms of heterogeneity are born.''~\citep[pg.\ 385]{condry:hopglob}
And even with the seemingly overwhelming popularity of American influences in mainstream Japanese culture, Japanese characteristics still prevail underneath the surface.  These observations in a culture with both great impetus and ample opportunity to Americanize paints a more optimistic picture of the future of cultural diversity in an age of globalization.  Cultures will never stay the same, but they can still remain distinct without strong geographic barriers separating them.  Furthermore, as we look at how Japanese culture has influenced popular culture in the United States, we will see that American culture itself is less monolithic than it may appear.

\section{A Uniform Melting Pot?}

The United States, ever since its relatively recent founding, has always been a nation composed of immigrants.  The traditional metaphor for the US has been the melting pot, where people from a variety of countries and backgrounds come together to form one uniform, shared American culture.  This culture resembled mainly middle-class protestant European influences, and while new waves of immigrants brought some changes, it remained relatively uniform throughout the 19th century.

While it is true that the United States is responsible for the opening of Japan and that the US maintained extensive trading relationships with Japan both before and after the war, US cultural borrowing from Japan has been limited. As \cSurface{\citet[pg.\ 47]} argues, ``Even if one considers the popularity of Japonism (that is, the initial, though often merely superficial, impact of Japanese art and architecture on the United States and Europe after Japan's opening to the West in the mid-nineteenth century) and the West's continued fascination with the remote and `esoteric' Orient, it is safe to say that Japanese culture has been more extensively shaped by its American counterpart than vice versa.  If it is true that Japan exports far more products and manufactured good than it imports, then it is also true that Japan imports vastly more information about or from the United States than the other way round.''  There are some famous examples of historical Japanese influences in the US, such as in the architecture of Frank Lloyd Wright; however, these were unusual cases, and had little effect on American culture as a whole.  After WWII, as Japan rebuilt and became an economic power, Americans began to worry about it from an economic standpoint, but there was still little influence flowing in the American direction.

\section{Commodities and Subcultures}

During Japan's economic boom, Japanese business practices were widely analyzed and espoused by American analysts and the efficient Japanese way was much prized for improving productivity.~\cCool{\citep[pg.\ 47]}  However, despite Japanese economic success, Japan had little visible effect on American popular culture beyond \emph{Godzilla}.  In fact, Japan did have a large impact on American lives, albeit indirectly: as \cRecentering{\citet[pg.\ 24]} writes, ``Japanese consumer technologies certainly have had a tremendous impact on our everyday life, an impact which is, in a sense, more profound than that of Hollywood films.'' 
%``Japanese consumer technologies certainly have had a tremendous impact on our everyday life, an impact which is, in a sense, more profound than that of Hollywood films. \ldots{} New cultural technologies open new possibilities for the consumption of media texts by audiences.''~\citep[pg.\ 24]{recentering} 
The walkman, the video game console, and the digital camera have all had huge impacts on American popular culture and the way we live.  However, none of these new opportunities really carries anything cultural about Japan with it.  So, unlike the case with American culture in Japan, while every American has an image of Japan and its products, this has not translated into the cultural spread that American products bring.

Logically, it seems that if American chains and product lines spread American culture worldwide, than the omnipresence of Japanese electronics and other products should spread Japanese traits in the same way.  However, while an American consumer may associate levels of quality or reliability with Japanese brands, there seems to be little connection of these products with any cultural values.  While certainly the technology that Japan is often associated with by Americans has affected many aspects of our day-to-day lives, it has done so through convenience, not through cultural change.  \cRecentering{\citet[pg.\ 26]} puts it, ``Despite the profound influence of Japanese consumer technologies on the cultural activities of our everyday life, they have tended not to be talked about in terms of a Japanese cultural presence.''  He continues by explaining that while the power and success associated with the US makes American culture an important part of the success of American exports, ``Japanese language is not widely spoken outside Japan and Japan is supposedly obsessed with its own cultural uniqueness.''
%``Despite the profound influence of Japanese consumer technologies on the cultural activities of our everyday life, they have tended not to be talked about in terms of a Japanese cultural presence. \ldots{} Cultural prestige, Western hegemony, the universal appeal of American popular culture, and the prevalence of the English language are no doubt advantageous to Hollywood.  By contrast, Japanese language is not widely spoken outside Japan and Japan is supposedly obsessed with its own cultural uniqueness.''~\citep[pg.\ 26]{recentering}
These differences help explain why Japan's economic strength has not translated to cultural power to anywhere near the extent that American power has.

One reason why Japanese products seem to carry little cultural influence is their perceived genericity.  The range of style in a car or camera is much more restricted than in a clothing line or a movie, and the clean functionality associated with Japanese products doesn't expand well into a greater cultural message.  Unlike a McDonald's hamburger or a Hollywood movie, ``The use of the Walkman does not evoke images or ideas of a Japanese lifestyle, even if consumers know it is made in Japan and appreciate `Japaneseness' in terms of its sophisticated technology.''~\cRecentering{\citep[pg.\ 28]}
%``However, no less important to the international success of McDonald's is its association with an attractive image of `the American way of life'\ldots{} The use of the Walkman does not evoke images or ideas of a Japanese lifestyle, even if consumers know it is made in Japan and appreciate `Japaneseness' in terms of its sophisticated technology.''~\citep[pg.\ 28]{recentering}
% characters in anim\'e nonracial
Even in Japanese animation and video games, arguably the vessels of Japanese popular culture with the greatest exposure overseas, the characters often look and feel much less Japanese than characters in Hollywood movies look American.  Traditionally, animated characters are drawn in a racially-ambiguous style that, while distinctive to Japanese artists, doesn't convey racial or cultural associations to foreign viewers.~\cRecentering{\citep[pg.\ 28]}  While the Japanese video game character Mario may be more well-known to Americans than Mickey Mouse, %needacite
the notionally Italian plumber is hardly associated with Japanese culture the way Mickey Mouse is connected with American culture.
%``The characters of Japanese animation and computer games for the most part do not look `Japanese.' \ldots{} Consumers of and audiences for Japanes animation and games, it can be argued, may be aware of the Japanese origin of these commodities, but those texts barely feature `Japanese body odor' identified as such.''~\citep[pg.\ 28]{recentering}
In addition, when Japanese shows and moves are imported for a mainstream American audience, they often deliberately lose Japanese culture references to make them more accessible to an American audience.  The classic example is the TV show Mighty Morphin' Power Rangers, which, although originally a Japanese show, was released in the US with all scenes featuring the characters outside their superhero costumes reshot with American actors, removing its potential to expose American audiences to Japanese culture.  Similarly, remakes of Japanese movies like \emph{Shall We Dance?} tend to be thoroughly Americanized.  Even more Japan-associated shows like Pok\'emon have references to Japanese foods and traditions replaced with American equivalents.  Thus, as \cRecentering{\citet[pg.\ 34]} explains, ``It is one thing to observe that Pok\'emon texts, for example, are influencing children's play and behavior in many parts of the world and that these children perceive Japan as a cool nation because it creates cool cultural products such as Pok\'emon.  However, it is quite another to say that this cultural influence and this perception of coolness is closely associated with a tangible, realistic appreciation of `Japanese' lifestyles or ideas.''  While Japan's economic strength have propelled its products worldwide, these are mostly devoid of the cultural connotations that have helped to drive Americanization.

%Conversely, cultural media stays at home
Along with this, Japanese media that does strongly reflect Japanese traditions tends to not make it successfully abroad.  As \cChing{\citet[pg.\ 171]} writes referring to the high-profile purchase of several Hollywood studios by Japanese corporations, ``Japan can buy Hollywood, but it is unable to produce films with global appeal.''  One reason for this is that Japanese filmmakers tend, quite understandably, to make films focused on the Japanese experience.  Without the prior interest in a foreign society that propels Hollywood films in Japan, films based on traditions, social issues, or environments specific to Japan seem uninteresting or hard to understand for American viewers.  In addition, while Japanese moviegoers are glad to watch undubbed Hollywood productions with Japanese subtitles, US audiences, in general, have little patience for subtitles.  This makes the cost of adapting a Japanese film to the American market much greater than the cost of releasing a Hollywood film in Japan.  Finally, the American familiarity and comfort with Hollywood movies and unfamiliarity with foreign films makes even high-quality Japanese films hard to pitch to US distributors. %need an example?
While the greater presence of Japanese and other Asian immigrants helps to counter these problems to some extent, their influence is primarily limited to their immediate community, having only a small effect on American culture as a whole.~\cPark{\citep[pg.\ 277]}
%``Especially, with the growing transnational (im)migration of Asians to the West and their transnational media consumption, the cultural flows between Asia and the West have been rapidly facilitated, reshaping the local sociocultural landscape of the West.''~\citep[pg.\ 277]{park}
Thus, more determined efforts are necessary in order for Japanese media to successfully bring cultural themes intact to the mainstream American audience.

However, in recent years, Japan has begun to move towards a greater spread of cultural ideas.  More artistic films like \emph{Akira}, which sold more copies in the US than in Japan \cEncyl{\citep[pg.\ 175]}, and the Academy Award-winning \emph{Spirited Away} have gained some mainstream exposure in the United States, though obviously no where near the exposure of American films in Japan.
One researcher writes, 
``Japanese animation and computer games have attained a certain degree of popularity and become recognized as very `Japanese' in a positive and affirmative sense in Western countries as well as in non-Western countries.  They embody a new aesthetic, emanating in large part from Japanese cultural inventiveness, and capture the new popular imagination.''~\cRecentering{\citep[pg.\ 30]}  One of the factors behind these movies' successes overseas is their high-quality English dubs.  Because of Americans' general dislike of subtitles, quality dubbing is important in order to reach a mainstream audience.  For example, the dubbing for \emph{Spirited Away} was handled by Disney, ensuring both that the dub would be well-done and that it would be accessible to Americans.  With this dub, Disney, instead of changing or simplifying the references to Japanese traditions that permeate the film, Disney added extra lines to briefly explain the necessary background so that viewers unfamiliar with Shint\=o spirits or the Japanese recession would not be confused by the plot.  Along with these movies, Japanese series targeted at older audiences have become more prevalent on American cable networks~\cCool{\citep[pg.\ 46]}, and these series as well increasingly preserve the cultural associations in the original works.  In contrast with child-targeted movies where Japanese culture references seldom survive, the increased exposure of these more artistic works allows elements of Japanese culture to reach mainstream audiences in a way it has been unable to previously.

% Fanatical subcultures
In addition to the small but growing mainstream exposure of Japanese animation, there is also a subculture among the American youth who take this art form much more seriously.  These \emph{otaku}, named after the Japanese equivalent for `nerd' and described by one scholar as ``obsessively devoted fans of Japanese animation \ldots{} whose craze for Japanese animation makes them wish they had been born in Japan''~\cRecentering{\citep[pg.\ 31]}, 
%``The emergence of obsessively devoted fans of Japanese animation in both Europe and the United States whose craze for Japanese animation makes them wish they had been born in Japan has been often covered by the Japanese media.''~\citep[pg.\ 31]{recentering}
are a small but visible subculture who use clubs, conventions, and the internet to share their love for and expand their knowledge of \emph{anime}, Japanese animation, and \emph{manga}, Japanese graphic novels.  Otaku go to great lengths to obtain a wide variety of anime and manga not widely available in the US, and also often study Japanese language, culture, and history to expand their understanding of these media.
One interesting practice among otaku is the sharing of `fan subs', pirated copies of anime shows and movies with fan-created English subtitles, distributed for free among otaku.  While at first glance this practice would seem disastrous to American sales of anime, in many cases the widespread distrbution of fan subs builds overseas popularity of a series or movie, creating a sufficient market for a successful professional English release.~\cPiracy{\citep[pg.\ 355]}
%``Anime thus provides another example of how permitting some copyright infringement has not caused the `fall of the anime industry', but may well have contributed to its worldwide popularity.''~\citep[pg.\ 355]{condry:piracy}
It is considered an ethical point among otaku to stop distributing fan subs for any series that has an official American release, creating a very different situation from the world of music piracy.  It is easy to dismiss otaku as an irrelevant subculture that has no significant effect on mainstream American culture. However, it is already large enough to enable Japanese anime and manga stores to open successful branches in the US~\cCool{\citep[pg.\ 46]}, and all signs indicate that it will continue to grow. The otaku movement is in fact a major factor that has contributed to growing exposure of authentic Japanese media, as discussed above, and also demonstrates that even the United States is not comprised of a uniform popular culture, but contains subgroups with diverse cultural interests unrelated to blood or geography.

The United States is used to being the cultural superpower, exporting its popular culture to the world seemingly without effort.  However, in an era of increasing global exposure, the melting pot isn't as uniform or as stable as it used to be.  While it is a much less than American influence is in Japan, Japanese influence is beginning to make itself be heard.  While its influence on the overall American popular culture is mostly limited to a small number of movies and TV series, their exposure and popularity are both increasing over time.  In addition, the small-but-visible youth subculture known as otaku are working not only to expand their own understanding of Japanese culture, but also to make in more available in the United States as a whole.  While Japan's popular products have not yet made it the cultural juggernaut that the US is worldwide, Japan is well on its way to greater influence.

\section{Conclusion}

Globalization opponents often paint the idea of a dystopian future where the world is composed of a single universal, materialistic culture, and all diversity and national distinctiveness has been lost.  While globalization certainly is having drastic effects on the cultural landscape of the world, evidence suggests that we are experiencing not a homogenization of culture, but simply a dramatic shift in the way cultures are identified, separated, and defined.
As \cHopGlob{\citet[pg.\ 384]} sees in Japan's hip-hop subculture, ``the construction of `who we are' arises increasingly from how we imagine ourselves, rather than from where we live.''
%``In other words, as identities can be picked up from a variety of media sources, the construction of `who we are' arises increasingly from how we imagine ourselves, rather than from where we live.''~\citep[pg.\ 384]{condry:hopglob}
This idea applies equally well to the otaku in America and to many other subgroups that have formed in the wake of better communication technologies and vanishing dividing lines
So, while cultures certainly change and influence each other as they always have, the end effect of globalization seems likely to be not a single uniform Americanized culture, but new, unique cultures arranged according to boundaries decided not by politicians but by the cultural participants, giving groups that would have been marginalized in the face of a uniform national culture a new voice. Rather than the uniformity globalization opponents fear, we face a new era with potentially even more, if differently arranged, cultural diversity.

%\section{Misc Quotes}

%``Especially, with the growing transnational (im)migration of Asians to the West and their transnational media consumption, the cultural flows between Asia and the West have been rapidly facilitated, reshaping the local sociocultural landscape of the West.''~\citep[pg.\ 277]{park}

%``In the late 1990s, the two cultures have thus entered a new phase of interactions.  Now chaotic and transculturally infectious negotiations occur between Orientalism and Occidentalism; between the Western belief in eternity and the Japanese aesthetics of the moment; between a Western productionist and idealist sensibility and a Japanese high-tech-consumerist and posthistorical mentality; or even between the science-fictional Japan of the American imagination and Japanese science fiction.''~\citep[pg.\ 231]{waiting}



\onehalfspacing

\bibliography{}

\end{document}